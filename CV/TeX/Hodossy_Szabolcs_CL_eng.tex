% !TXS template
\documentclass[11pt,a4paper,sans,english]{moderncv}        % possible options include font size ('10pt', '11pt' and '12pt'), paper size ('a4paper', 'letterpaper', 'a5paper', 'legalpaper', 'executivepaper' and 'landscape') and font family ('sans' and 'roman')
\moderncvstyle{casual}                             % style options are 'casual' (default), 'classic', 'oldstyle' and 'banking'
\moderncvcolor{blue}                               % color options 'blue' (default), 'orange', 'green', 'red', 'purple', 'grey' and 'black'
%\nopagenumbers{}                                  % uncomment to suppress automatic page numbering for CVs longer than one page
\usepackage[utf8]{inputenc}                       % if you are not using xelatex ou lualatex, replace by the encoding you are using
\usepackage[scale=0.75,a4paper]{geometry}
\usepackage{babel}
%----------------------------------------------------------------------------------
%            personal data
%----------------------------------------------------------------------------------
\firstname{Szabolcs}%<first name%:columnShift:-1,persistent%>
\familyname{Hodossy}%<family name%:columnShift:-1,persistent%>
%\title{}%<Resumé title%:columnShift:-1,persistent%>                               % optional, remove/comment the line if not wanted
\address{Fáy köz 8. I/1}{1035 Budapest}{Hungary}%<street and number%:columnShift:-5,persistent%>%<postcode city%:columnShift:-3,persistent%>%<country%:columnShift:-1,persistent%>         % optional, remove/comment the line if not wanted; the "country" arguments can be omitted or provided empty
\mobile{+36203133814}%<mobile number%:columnShift:-1,persistent%>                          % optional, remove/comment the line if not wanted
%\phone{}%<phone number%:columnShift:-1,persistent%>                           % optional, remove/comment the line if not wanted
%\fax{}%<fax number%:columnShift:-1,persistent%>                             % optional, remove/comment the line if not wanted
\email{hodossy.szabolcs@gmail.com}%<email%:columnShift:-1,persistent%>                               % optional, remove/comment the line if not wanted
\homepage{https://hu.linkedin.com/in/szabolcs-hodossy-228b32116}%<home page%:columnShift:-1,persistent%>                         % optional, remove/comment the line if not wanted
%\extrainfo{}%<additional information%:columnShift:-1,persistent%>                 % optional, remove/comment the line if not wanted
\photo[64pt][0.4pt]{kiskep3.jpg}%<picture%:columnShift:-1,persistent%>                       % optional, uncomment the line if wanted; '64pt' is the height the picture must be resized to, 0.4pt is the thickness of the frame around it (put it to 0pt for no frame) and 'picture' is the name of the picture file
%\quote{}%<some quote%:columnShift:-1,persistent%>                                 % optional, remove/comment the line if not wanted
%
\begin{document}
\clearpage
%-----       letter       ---------------------------------------------------------
% recipient data
\recipient{HR Department}{Ericsson}%<Company Recruitment team%:columnShift:-7,persistent%>%<Company's name%:columnShift:-5,persistent%>%<Street address%:columnShift:-3,persistent%>%<Zip Code City%:columnShift:-1,persistent%>
\date{\today}%<Date%:columnShift:-1,persistent%>
\opening{Dear Sir or Madam,}%<Dear Sir or Madam,%:columnShift:-1,persistent%>
\closing{Yours faithfully,}%<Yours faithfully,%:columnShift:-1,persistent%>
\enclosure{Curriculum Vitae}%<enclosures%:columnShift:-1,persistent%>          % use an optional argument to use a string other than "Enclosure", or redefine \enclname
\makelettertitle
I am a Data Engineer who has actually never worked on classical data engineering problems, but Fullstack Web Development has always been part of my job. At Continental, I had the opportunity and responsibility to architect a complete web application, from choosing a tech stack to defining the architecture and leading the development of the project. Therefore I have decided to make that carrier switch "official" and work as a Web Developer.

At Nokia I have seen lot's of legacy code and learned a lot from mistakes made by myself and others in the past. Then I had the opportunity to start a green field project still at Nokia and then at Continental. During those projects I had the opportunity to experiment with different architectures and refine my web developer skills. I turned to open source solutions and since I have become an enthusiast. What I like the most in open source communities is the transparency, the positive attitude in which they treat each other and the ability to coordinate the work of developers across the globe. I am looking forward to something similar as an employee as well.

I also had the opportunity to author some Python packages and give back a little to the community I have learnt so much from. At Nokia I got the permission to release a small part of my project as an OSS called Pandas Extras (pandas-extras), and during my work at Continental I have seen a particular problem which I was able to address, this time in my own free time, as a package called Django Natural Language Filter (django-nlf). These packages, while not so popular yet, are a statement of my work on the backend. I am still looking forward the opportunity to author an Angular package, but I have a small web app to show my frontend work as well, called Whostarts? and available at hodossy.github.io/whostarts/.

I think the opening of an Experienced Javascript/Python Developer is the perfect next step for me to take. I believe that this combination of knowledge and wide range of experience is valuable for You, and can make me a strong candidate for this job.

I hope I managed to make You interested in me as a successful applicant for the job. I am looking forward your answer!

\makeletterclosing
\end{document}
